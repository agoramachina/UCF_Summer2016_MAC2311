\documentclass[fleqn]{article}
\usepackage{geometry, mathtools, bm, amssymb, tabularx, pbox, enumitem}
\geometry{portrait, margin=.75in}
\pagestyle{empty}
\begin{document}

\Huge\textbf{Chapter 2}
\vspace{16pt}

\begin{center}
\Large\textbf{2.1 The Tangent and Velocity Problems}

\noindent\hfill\rule{0.3\textwidth}{.4pt}\hfill
\vspace{12pt}

\large
\def\arraystretch{1.3}
{\setlength{\tabcolsep}{16pt}
\begin{tabularx}{.9\textwidth}{|X|}
\hline
	\vspace{1pt}
	\textbf{The Tangent Problem} \\[5pt]
	We can find the equation of the tangent line $t$ as soon as we know its slope $m$. We need two points to compute the slope, but with one point $P$ we can compute an approximation to $m$ by choosing a nearby point $Q(a, f(a))$ and computing the slope $m_{PQ}$ of the secant line $PQ$ \\[5pt]
	The slope of the tangent line is the \textit{limit} of the slopes of the secant lines, expressed as 
	$$\underset{P \to Q}{\lim} \: m_{PQ} = m$$
	Then, we use the point-slope form of the equation of a line to write the equation of the tangent line through $(x, y)$ as
	$$y - y_1 = m(x-x_1)$$
	This can also be represented as
	$$m_{PQ} = \frac{f(x)-f(a)}{x-a} \text{\hspace{32pt} or \hspace{32pt}} m = \underset{x \to a}{\lim}\frac{f(x)-f(a)}{x-a}$$ \\	
\hline
\end{tabularx}}
\vspace{12pt}

\def\arraystretch{1.3}
{\setlength{\tabcolsep}{16pt}
\begin{tabularx}{.9\textwidth}{|X|}
\hline
	\vspace{1pt}
	\textbf{The Velocity Problem} \\[5pt]
	There is a close connection between the tangent problem and the problem of finding velocities. If we consider the points $P(a, f(a))$ and $Q(a+h, f(a+h))$, the slope of the secant line $PQ$ is the same as the average velocity over the time interval $[a, a+h]$. Therefore the velocity at time $t=a$ (the limit of these average velocities as $h$ approaches $0$) must be equal to the slope of the tangent line at $P$ (the limit of the slopes of the secant lines).
	$$\text{average velocity} = \dfrac{\text{change in position}}{\text{time elapsed}}$$
	The \textbf{average velocity} is equal to the slope of the secant line
		$$\bar{v} = \frac{f(a+h)-f(a)}{h}\text{\hspace{32pt} or \hspace{32pt}} \bar{v} = \dfrac{\Delta s}{\Delta t}$$
	The \textbf{instantaneous velocity} is equal to the slope of the tangent line 
		$$v = \lim_{h \to 0} \frac{f(a+h)-f(a)}{h} \text{\hspace{32pt} or \hspace{32pt}} v = \dfrac{ds}{dt}$$ \\

\hline
\end{tabularx}}
\vspace{12pt}
\pagebreak

\Large\textbf{2.2 The Limit of a Function}

\noindent\hfill\rule{0.3\textwidth}{.4pt}\hfill
\vspace{12pt}

\large
\def\arraystretch{1.3}
{\setlength{\tabcolsep}{16pt}
\begin{tabularx}{.9\textwidth}{|X|}
\hline
	\vspace{1pt}
	\textbf{Definition of Limits} \\[5pt]
	Suppose $f(x)$ is defined when $x$ is near the number $a$ (this means that $f$ is defined on some open interval that contains $a$, except possibly at $a$ itself).  Then we write
	$$\underset{x \to a}{\lim} f(x) = L$$
	which is read as "$f(x)$ approaches $L$ as $x$ approaches $a$".  \\[16pt]
\hline
\end{tabularx}}
\vspace{12pt}

\def\arraystretch{1.3}
{\setlength{\tabcolsep}{16pt}
\begin{tabularx}{.9\textwidth}{|X|}
\hline
	\vspace{1pt}
	\textbf{Left and Right Hand Limits} \\[5pt]
	The left-hand limit of $f(x)$ as $x$ approaches $a$ (or, the limit of $f(x)$ as $x$ approaches $a$ from the left) is equal to $L$ if we can make the values of $f(x)$ arbitrarily close to $L$ by taking $x$ to be sufficiently close to $a$ and $x < a$.
	$$\underset{x \to a^-}{\lim} f(x) = L$$
	Right hand limits are represented as
	$$\underset{x \to a^+}{\lim} f(x) = L$$ \\
\hline
\end{tabularx}}
\vspace{12pt}

\def\arraystretch{1.3}
{\setlength{\tabcolsep}{16pt}
\begin{tabularx}{.9\textwidth}{|X|}
\hline
	\vspace{1pt}
	By combining the definitions of one-sided and two-sided limits, we see that
	$$\underset{x \to a}{\lim} f(x) = L \text{\hspace{12pt} if and only if \hspace{12pt}} \underset{x \to a^-}{\lim} f(x) = L \text{\hspace{12pt} and \hspace{12pt}} \underset{x \to a^+}{\lim} f(x) = L$$ \\
\hline
\end{tabularx}}
\vspace{12pt}

\def\arraystretch{1.3}
{\setlength{\tabcolsep}{16pt}
\begin{tabularx}{.9\textwidth}{|X|}
\hline
	\vspace{1pt}
	\textbf{Infinite Limits} \\
	Let $f$ be a function defined on both sides of $a$, except possibly at $a$ itself. Then
	$$\underset{x \to a}{\lim} f(x) = \infty$$
	$$\underset{x \to a}{\lim} f(x) = -\infty$$
	means that the values of $f(x)$ can be made arbitrarily large(positive) or large(negative) by taking $x$ sufficiently close, but not equal to $a$. \\[5pt]
	(Note that the symbol $\infty$ does not represent a number, and thus the Limit Laws cannot be applied to infinite limits)\\[16pt]
\hline
\end{tabularx}}
\vspace{12pt}

\def\arraystretch{1.3}
{\setlength{\tabcolsep}{16pt}
\begin{tabularx}{.9\textwidth}{|>{\centering}X >{\centering}X >{\centering\arraybackslash}X|}
\hline
	\vspace{1pt} 
	\pbox{20cm}{\textbf{Vertical Asymptotes}} & & \\
	\multicolumn{3}{|>{\hsize=\dimexpr3\hsize+4\tabcolsep+2\arrayrulewidth\relax}X|}{\vspace{-5pt}The line $x=a$ is called a \textbf{vertical asymptote} of the curve $y=f(x)$ if at least one of the following statements is true:} \\
	\vspace{-12pt} & \vspace{-12pt} & \vspace{-12pt}\\[-12pt]
	$\underset{x \to a}{\lim} f(x) = \infty$ & $\underset{x \to a^-}{\lim} f(x) = \infty$ & $\underset{x \to a^+}{\lim} f(x) = \infty$ \\[12pt]
	$\underset{x \to a}{\lim} f(x) = -\infty$ & $\underset{x \to a^-}{\lim} f(x) = -\infty$ & $\underset{x \to a^+}{\lim} f(x) = -\infty$ \\[16pt]	
\hline
\end{tabularx}}
\vspace{32pt}

\Large\textbf{2.3 Calculating Limits Using the Limit Laws}

\noindent\hfill\rule{0.3\textwidth}{.4pt}\hfill
\vspace{12pt}

\large
\def\arraystretch{1.3}
{\setlength{\tabcolsep}{16pt}
\begin{tabularx}{.9\textwidth}{|X|}
\hline
	\vspace{1pt}
	\textbf{Limit Laws} \\
	Suppose that $c$ is a constant and these limits exist
	$$\underset{x \to a}{\lim} \: f(x) \text{\hspace{12pt} and \hspace{12pt}} \underset{x \to a}{\lim} \: g(x)$$
	Then
	\[\text{\textbf{1. }} \underset{x \to a}{\lim} [f(x) + g(x)] = \underset{x \to a}{\lim} \: f(x) + \underset{x \to a}{\lim} \: g(x)\]
	\[\text{\textbf{2. }} \underset{x \to a}{\lim} [f(x) - g(x)] = \underset{x \to a}{\lim} \: f(x) - \underset{x \to a}{\lim} \: g(x)\] 
	\[\text{\textbf{3. }} \underset{x \to a}{\lim} \: [c \: f(x)] = c \: \underset{x \to a}{\lim} \: f(x) \] 
	\[\text{\textbf{4. }} \underset{x \to a}{\lim} [f(x) \: g(x)] = \underset{x \to a}{\lim} \: f(x) \cdot \underset{x \to a}{\lim} \: g(x)\] \\[-32pt] 
	\[\text{\textbf{5. }} \underset{x \to a}{\lim} \dfrac{f(x)}{g(x)} = \dfrac{\underset{x \to a}{\lim} \: f(x)}{\underset{x \to a}{\lim} \: g(x)} \text{\hspace{12pt} if \hspace{12pt}} \underset{x \to a}{\lim} \: g(x) \ne 0 \] 
	\[\text{\textbf{6. }} \underset{x \to a}{\lim} [f(x)]^n = \Big[\underset{x \to a}{\lim} f(x) \Big]^n \] \\[-32pt]
	\[\text{\textbf{7. }} \underset{x \to a}{\lim} \: c = c \] 
	\[\text{\textbf{8. }} \underset{x \to a}{\lim} \: x = a \] 
	\[\text{\textbf{9. }} \underset{x \to a}{\lim} \: x^n = a^n \text{\hspace{30pt} where \hspace{12pt} $n$ is a positive integer}\] 
	\[\text{\textbf{10. }} \underset{x \to a}{\lim} \: \sqrt[n]{x} = \sqrt[n]{a} \text{\hspace{12pt} where \hspace{12pt} $n$ is a positive integer}\] 
	\hspace{135pt}(if $n$ is even, assume that $a > 0$)
	\[\text{\textbf{11. }} \underset{x \to a}{\lim} \: \sqrt[n]{f(x)} = \sqrt[n]{\underset{x \to a}{\lim} \: f(x)} \text{\hspace{12pt} where \hspace{12pt} $n$ is a positive integer}\] 
	\hspace{195pt}(if $n$ is even, assume that $a > 0$) \\
\hline
\end{tabularx}}
\vspace{12pt}

\def\arraystretch{1.3}
{\setlength{\tabcolsep}{16pt}
\begin{tabularx}{.9\textwidth}{|X|}
\hline
	\vspace{1pt}
	\textbf{Direct Substitution Property} \\
	If $f$ is a polynomial or a rational function and $a$ is in the domain of $f$, then
	$$\underset{x \to a}{\lim} \: f(x) = f(a)$$ \\
\hline
\end{tabularx}}
\vspace{12pt}

\def\arraystretch{1.3}
{\setlength{\tabcolsep}{16pt}
\begin{tabularx}{.9\textwidth}{|X|}
\hline
	\vspace{1pt}
	If $f(x) = g(x)$ when $x \ne a$, then \hspace{12pt} $\underset{x \to a}{\lim} \: f(x) = g(x)$ \hspace{12pt} provided the limits exist \\[16pt]
\hline
\end{tabularx}}
\vspace{12pt}

\def\arraystretch{1.3}
{\setlength{\tabcolsep}{16pt}
\begin{tabularx}{.9\textwidth}{|X|}
\hline
	\vspace{1pt}
	$\underset{x \to a}{\lim} \: f(x) = L$ \hspace{12pt} if and only if \hspace{12pt} $\underset{x \to a^-}{\lim} \: f(x) = L = \underset{x \to a^+}{\lim} \: f(x)$ \\[16pt]
\hline
\end{tabularx}}
\vspace{12pt}

\def\arraystretch{1.3}
{\setlength{\tabcolsep}{16pt}
\begin{tabularx}{.9\textwidth}{|X|}
\hline
	\vspace{1pt}
	If $f(x) \le g(x)$ when $x$ is near $a$ (except possibly at $a$) and the limits of $f$ and $g$ both exist as $x$ approaches $a$, then
	$$\underset{x \to a}{\lim} \: f(x) \le \underset{x \to a}{\lim} \: g(x)$$ \\
\hline
\end{tabularx}}
\vspace{12pt}

\def\arraystretch{1.3}
{\setlength{\tabcolsep}{16pt}
\begin{tabularx}{.9\textwidth}{|X|}
\hline
	\vspace{1pt}
	\textbf{The Squeeze Theorem} \\
	If $f(x) \le g(x) \le h(x)$ when $x$ is near $a$ (except possibly at $a$) and
	$$\underset{x \to a}{\lim} \: f(x) = \underset{x \to a}{\lim} \: h(x) = L$$
	then
	$$\underset{x \to a}{\lim} \: g(x) = L$$ \\
\hline
\end{tabularx}}
\vspace{32pt}
\pagebreak

\Large\textbf{2.5 Continuity}

\noindent\hfill\rule{0.3\textwidth}{.4pt}\hfill
\vspace{12pt}

\large
\def\arraystretch{1.3}
{\setlength{\tabcolsep}{16pt}
\begin{tabularx}{.9\textwidth}{|X|}
\hline
	\vspace{1pt}
	A function $f$ is \textbf{continuous} at a number $a$ if
	$$\underset{x \to a}{\lim} \: f(x) = f(a)$$ 
	This implicitly requires three things if $f$ is continuous at $a$: \\
	\textbf{1. } $f(a)$ is defined (that is, $a$ is in the domain of $f$) \\
	\textbf{2. } $\underset{x \to a}{\lim} \: f(x)$ exists \\
	\textbf{3. } $\underset{x \to a}{\lim} \: f(x) = f(a)$ \\[12pt]
	We say that $f$ has a \textbf{discontinuity} at $a$ if $f$ is not continuous at $a$ \\[16pt]
\hline
\end{tabularx}}
\vspace{12pt}

\def\arraystretch{1.3}
{\setlength{\tabcolsep}{16pt}
\begin{tabularx}{.9\textwidth}{|X|}
\hline
	\vspace{1pt}
	A function $f$ is \textbf{continuous from the right} at a number $a$ if 
	$$\underset{x \to a^+}{\lim} \: f(x) = f(a)$$
	and $f$ is \textbf{continuous from the left} at $a$ if
	$$\underset{x \to a^-}{\lim} \: f(x) = f(a)$$ \\
\hline
\end{tabularx}}
\vspace{12pt}

\def\arraystretch{1.3}
{\setlength{\tabcolsep}{16pt}
\begin{tabularx}{.9\textwidth}{|X|}
\hline
	\vspace{1pt}
	A function $f$ is \textbf{continuous on an interval} if it is continuous at every  number in the interval. (If $f$ is defined only on one side of an endpoint of the interval, we understand \textit{continuous} at the endpoint to mean \textit{continous from the right} or \textit{continuous from the left}). \\[16pt]
\hline
\end{tabularx}}
\vspace{12pt}

\def\arraystretch{1.3}
{\setlength{\tabcolsep}{16pt}
\begin{tabularx}{.9\textwidth}{|X X X|}
\hline
	\multicolumn{3}{|>{\hsize=\dimexpr3\hsize+4\tabcolsep+2\arrayrulewidth\relax}X|}{\vspace{-5pt}If $f$ and $g$ are continuous at $a$ and $c$ is a constant, then the following functions are also continuous at $a$:} \\
	\vspace{-12pt} & \vspace{-12pt} & \vspace{-12pt}\\[-12pt]
	$\text{\textbf{1. }} f + g$ & $\text{\textbf{2. }} f - g$ & $\text{\textbf{3. }} cf$ \\
	$\text{\textbf{4. }} f g$ & $\text{\textbf{5. }} \dfrac{f}{g} \:\:$ if $g(a) \ne 0$ & \\[16pt] 
\hline
\end{tabularx}}
\vspace{12pt}

%\vspace*{\fill}
\def\arraystretch{1.3}
{\setlength{\tabcolsep}{16pt}
\begin{tabularx}{.9\textwidth}{|X|}
\hline
	\begin{enumerate}[label=\textbf{\alph*}.]
	\item Any polynomial is continuous everywhere; that is, it is continuous on \hspace{100pt} $\mathbb{R} = (-\infty, \infty)$. 
	\item Any rational function is continuous wherever it is defined; that is, it is continuous on its domain.
	\end{enumerate}
	\\[5pt]
\hline
\end{tabularx}}
\vspace{12pt}

\def\arraystretch{1.3}
{\setlength{\tabcolsep}{16pt}
\begin{tabularx}{.9\textwidth}{|X X X|}
\hline 
	\multicolumn{3}{|>{\hsize=\dimexpr3\hsize+4\tabcolsep+2\arrayrulewidth\relax}X|}{\vspace{-5pt} The following types of functions are continuous at every number in their domains:} \\
	\vspace{-12pt} & \vspace{-12pt} & \vspace{-12pt}\\[-12pt]
	polynomials & rational functions & root functions \\[3pt]
	trigonometric functions & inverse trig functions & \\[3pt]
	exponential functions & logarithmic functions & \\[16pt] 
\hline
\end{tabularx}}
\vspace{12pt}

\def\arraystretch{1.3}
{\setlength{\tabcolsep}{16pt}
\begin{tabularx}{.9\textwidth}{|X|}
\hline
	\vspace{1pt}
	If $f$ is continuous at $b$ and $\underset{x \to a}{\lim} \: g(x) = b$, then $\underset{x \to a}{\lim} \: f(g(x)) = f(b)$. \\
	In other words,
	$$\underset{x \to a}{\lim} \: f(g(x)) = f(\underset{x \to a}{\lim} \: g(x))$$ \\
\hline
\end{tabularx}}
\vspace{12pt}

\def\arraystretch{1.3}
{\setlength{\tabcolsep}{16pt}
\begin{tabularx}{.9\textwidth}{|X|}
\hline
	\vspace{1pt}
	If $g$ is continuous at $a$ and $f$ is continuous at $g(a)$, then the composite function $f \circ g$ given by $(f \circ g)(x) = f(g(x))$ is continuous at $a$. \\[16pt]
	\hline
\end{tabularx}}
\vspace{12pt}

\def\arraystretch{1.3}
{\setlength{\tabcolsep}{16pt}
\begin{tabularx}{.9\textwidth}{|X|}
\hline
	\vspace{1pt}
	\textbf{The Intermediate Value Theorem} \\
	Suppose that $f$ is continuous on the closed interval $[a, b]$ and let $N$ be any number between $f(a)$ and $f(b)$, where $f(a) \ne f(b)$. Then there exists a number $c$ in $(a, b)$ such that $f(c) = N$. \\[16pt]
	\hline
\end{tabularx}}
\vspace{32pt}
%\vspace*{\fill}

\pagebreak

\Large\textbf{2.6 Limits at Infinity; Horizontal Asymptotes}

\noindent\hfill\rule{0.3\textwidth}{.4pt}\hfill
\vspace{12pt}

\large
\def\arraystretch{1.3}
{\setlength{\tabcolsep}{16pt}
\begin{tabularx}{.9\textwidth}{|X|}
\hline
	\vspace{1pt}
	Let $f$ be a function defined on some interval $(a, \infty)$. Then 
	$$\underset{x \to \infty}{\lim} \: f(x) = L$$
	means that the values of $f(x)$ can be made arbitrarily close to $L$ by taking $x$ sufficiently large. \\[5pt]
	Similarly, let $f$ be a function defined on some interval $(a, -\infty)$. Then 
	$$\underset{x \to -\infty}{\lim} \: f(x) = L$$
	means that the values of $f(x)$ can be made arbitrarily close to $L$ by taking $x$ sufficiently large negative. \\[16pt]
	\hline
\end{tabularx}}
\vspace{12pt}

\def\arraystretch{1.3}
{\setlength{\tabcolsep}{16pt}
\begin{tabularx}{.9\textwidth}{|X|}
\hline
	\vspace{1pt}
	The line $y = L$ is called a \textbf{horizontal asymptote} of the curve $y = f(x)$ if either
	$$\underset{x \to \infty}{\lim} \: f(x) = L \text{\hspace{12pt} or \hspace{12pt}} \underset{x \to -\infty}{\lim} \: f(x) = L$$
	\\
	\hline
\end{tabularx}}
\vspace{12pt}

\def\arraystretch{1.3}
{\setlength{\tabcolsep}{16pt}
\begin{tabularx}{.9\textwidth}{|X|}
\hline
	\vspace{1pt}
	If $r > 0$ is a rational number, then 
	$$\underset{x \to \infty}{\lim} \dfrac{1}{x^r} = 0$$
	If $r > 0$ is a rational number such that $x^r$ is defined for all $x$, then
	$$\underset{x \to -\infty}{\lim} \dfrac{1}{x^r} = 0$$	
	\\
	\hline
\end{tabularx}}
\vspace{12pt}

\def\arraystretch{1.3}
{\setlength{\tabcolsep}{16pt}
\begin{tabularx}{.9\textwidth}{|X|}
\hline
	\vspace{1pt}
	\textbf{Examples} \\
	An example of a curve with two horizontal asymptotes is $y = tan^{-1} x$
	$$\underset{x \to -\infty}{\lim} tan^{-1} x = -\dfrac{\pi}{2} \hspace{24pt} \underset{x \to \infty}{\lim} tan^{-1} x = \dfrac{\pi}{2}$$
	The graph of the natural exponential function $y = e^x$ has the line $y=0$ (the $x$-axis) as a horizontal asymptote.
	$$\underset{x \to -\infty}{\lim} e^x = 0$$
	\\
	\hline
\end{tabularx}}
\vspace{12pt}

\Large\textbf{2.7 Derivatives and Rates of Change}

\noindent\hfill\rule{0.3\textwidth}{.4pt}\hfill
\vspace{12pt}

\large
\def\arraystretch{1.3}
{\setlength{\tabcolsep}{16pt}
\begin{tabularx}{.9\textwidth}{|X|}
\hline
	\vspace{1pt}
	\textbf{Tangents} \\
	If a curve $C$ has equation $y = f(x)$ and we want to find the tangent line to $C$ at the point $P(a, f(a))$, then we consider a nearby point $Q(x, f(x))$ where $x \ne a$, and compute the slope of the secant line $PQ$. Then we let $Q$ approach $P$ along the curve $C$ by letting $x$ approach $a$. If $m_{PQ}$ approaches a number $m$, then we define the \textit{tangent} $t$ to be the line through $P$ with slope $m$. This amounts to saying that the tangent line is the limiting position of the secant line $PQ$ as $Q$ approaches $P$.
	$$m = \underset{x \to a}{\lim} \dfrac{f(x) - f(a)}{x-a}$$
	There is another expression for the slope of a tangent line that is sometimes easier to use. If $h=x-a$, then $x=a+h$ and so the slope of the secant line $PQ$ is 
	$$m = \underset{h \to 0}{\lim} \dfrac{f(a+h) - f(a)}{h}$$
	\\
	\hline
\end{tabularx}}
\vspace{12pt}

\def\arraystretch{1.3}
{\setlength{\tabcolsep}{16pt}
\begin{tabularx}{.9\textwidth}{|X|}
\hline
	\vspace{1pt}
	\textbf{Velocity} \\
	Suppose an object moves along a straight line according to an equation of motion $s = f(t)$, where $s$ is the displacement (directed distance) of the object from the origin at time $t$. The function $f$ that describes the motion is called the \textbf{position function} of the object. In the time interval from $t=a$ to $t=a+h$ the change in position is $f(a+h) - f(a)$. The average velocity over this time interval is
	$$\text{average velocity} = \dfrac{\text{displacement}}{\text{time}} = \dfrac{f(a+h) - f(a)}{h}$$
	which is the same as the slope of the secant line $PQ$. \\
	Now suppose we compute the average velocities over shorter and shorter time intervals $[a, a+h]$. In other words, we let $h$ approach $0$. We define the \textbf{velocity} (or \textbf{instantaneous velocity} $v(a)$ at time $t=a$ to be the limit of these average velocities
	$$v(a) = \underset{h \to 0}{\lim} \dfrac{f(a+h) - f(a)}{h}$$
	This means that the velocity at time $t=a$ is equal to the slope of the tangent line at $P$. 
	\\[16pt]
	\hline
\end{tabularx}}
\vspace{12pt}

\def\arraystretch{1.3}
{\setlength{\tabcolsep}{16pt}
\begin{tabularx}{.9\textwidth}{|X|}
\hline
	\vspace{1pt}
	\textbf{Derivatives} \\
	The \textbf{derivative of a function $f$ at a number $a$}, denoted by $f'(a)$, is
	$$f'(a) = \underset{h \to 0}{\lim} \dfrac{f(a+h) - f(a)}{h}$$
	or equivalently
	$$f'(a) = \underset{x \to a}{\lim} \dfrac{f(x) - f(a)}{x-a}$$
	if this limit exists. \\
	In other words, the tangent line to $y=f(x)$ at $(a, f(a))$ is the line through $(a, f(a))$ whose slope is equal to $f'(a)$, the derivative of $f$ at $a$.
	\\[16pt]
	\hline
\end{tabularx}}
\vspace{32pt}

\Large\textbf{2.8 The Derivative as a Function}

\noindent\hfill\rule{0.3\textwidth}{.4pt}\hfill
\vspace{12pt}

\large
\def\arraystretch{1.3}
{\setlength{\tabcolsep}{16pt}
\begin{tabularx}{.9\textwidth}{|X|}
\hline
	\vspace{1pt}
	Instead of considering the derivative of function $f$ at a fixed number $a$
	$$f'(a) = \underset{h \to 0}{\lim} \dfrac{f(a+h) - f(a)}{h}$$
	by replacing $a$ by a variable $x$ we obtain
	$$f'(x) = \underset{h \to 0}{\lim} \dfrac{f(x+h) - f(x)}{h}$$
	We can regard $f'$ as a new function, called the \textbf{derivative of $f$} and defined that this equation, and we know that the value of $f'$ at $x$, $f'(x)$ can be interpreted geometrically as the slope of the tangent line to the graph of $f$ at the point $(x, f(x))$.\\[5pt]
	The domain of $f'$ is the set $\{x \: | \: f'(x) \text{exists}\}$ and may be smaller than the domain of $f$.
	\\[16pt]
	\hline
\end{tabularx}}
\vspace{12pt}

\def\arraystretch{1.3}
{\setlength{\tabcolsep}{16pt}
\begin{tabularx}{.9\textwidth}{|X|}
\hline
	\vspace{1pt}
	\textbf{Leibniz Notation} \\
	If we use the traditional notation $y=f(x)$ to indicate that the independent variable is $x$ and the dependent variable is $y$, then alternative notations for the derivative are as follows:
	$$f'(x) = y' = \dfrac{dy}{dx} = \dfrac{df}{dx} = \dfrac{d}{dx} f(x) = D \: f(x) = D_x f(x)$$
	The symbols $D$ and $d/dx$ are \textbf{differention operators} because they indicate the operation of \textbf{differention}. It is especially useful when used in conjunction with increment notation
	$$\dfrac{dy}{dx} = \underset{\Delta x \to 0}{\lim} \dfrac{\Delta y}{\Delta x}$$
	If we want to indicate the value of a derivative $dy/dx$ in Leibniz notation at a specific number $a$, we use the notation
	$$\dfrac{dy}{dx} \Big|_{x \to a} \hspace{24pt} \text{or} \hspace{24pt} \dfrac{dy}{dx} \Big]_{x \to a}$$
	which is a synonym for $f'(a)$
	\\[16pt]
	\hline
\end{tabularx}}
\vspace{12pt}

\def\arraystretch{1.3}
{\setlength{\tabcolsep}{16pt}
\begin{tabularx}{.9\textwidth}{|X|}
\hline
	\vspace{1pt}
	A function $f$ is \textbf{differentiable at $a$} if $f'(a)$ exists. It is \textbf{differentiable on an open interval} $(a, b)$ [or $(a, \infty)$ or $(-\infty, a)$ or $(-\infty, \infty)$] if it is differentiable at every number in the interval. \\[5pt]
	If $f$ is differentiable at $a$, then $f$ is continuous at $a$.  The converse is false; that is, there are functions that are continuous but not differentiable (such as the Weierstrass function).
	\\[16pt]
	\hline
\end{tabularx}}
\vspace{12pt}

\def\arraystretch{1.3}
{\setlength{\tabcolsep}{16pt}
\begin{tabularx}{.9\textwidth}{|X|}
\hline
	\vspace{1pt}
	\textbf{How can a Function Fail to be Differentiable?} \\
	In general, if the graph of a function $f$ has a "corner" or "kink" in it, then the graph of $f$ has no tangent at this point and $f$ is not differentiable there (in trying to compute $f'(a)$, we find that the left and right limits are different). \\[3pt]
	A second possibility is that if $f$ is not continuous at $a$, then $f$ is not differentiable at $a$. \\[3pt]
	A third possibility is that the curve has a \textbf{vertical tangent line} when $x = a$; that is, $f$ is continuous at $a$ and 
	$$\underset{x \to a}{\lim} |f'(x)| = \infty$$
	This means that the tangent lines become steeper and steeper as $x \to a$.
	\\[16pt]
	\hline
\end{tabularx}}
\vspace{12pt}

\def\arraystretch{1.3}
{\setlength{\tabcolsep}{16pt}
\begin{tabularx}{.9\textwidth}{|X|}
\hline
	\vspace{1pt}
	\textbf{Higher Derivatives} \\
	If $f$ is a differentiable function, then its derivative $f'$ is also a function, so $f'$ may have a derivative of its own, denoted by $(f')' = f''$. This new function $f''$ is called the \textbf{second derivative} of $f$. Using Leibniz notation, we write the second derivative of $y = f(x)$ as
	$$\dfrac{d}{dx} \Big( \dfrac{dy}{dx} \Big) = \dfrac{d^2 y}{dx^2}$$
	We can interpret $f''(x)$ as the slope of the curve $y= f'(x)$ at the point $(x, f'(x))$. In other words, it is the rate of change of the slope of the original curve $y = f(x)$. \\[5pt]
	\textbf{Note that $f''(x)$ is negative when $y = f'(x)$ has negative slope and positive when $y = f'(x)$ has positive slope.} \\[5pt]
	In general, we can interpret a second derivative as a rate of change of a rate of change. The most familiar example of this is \textit{acceleration}, which we define as follows:\\[3pt]
	If $s(t)$ is the position function of an object that moves in a straight line, we know that its first derivative represents the velocity $v(t)$ of the object as a function of time:
	$$v(t) = s'(t) = \dfrac{ds}{dt}$$
	The instantaneous rate of change of velocity with respect to time is called the \textbf{acceleration} $a(t)$ of the object. Thus the acceleration function is the derivative of the velocity function and is therefore the second derivative of the position function:
	$$a(t) = v'(t) = s''(t)$$
	or, in Leibniz notation, 
	$$a = \dfrac{dv}{dt} = \dfrac{d^2 s}{dt^2}$$
	The \textbf{third derivative} $f'''$ is the derivative of the second derivative: $f''' = (f'')'$. So $f'''(x)$ can be interpreted as the slope of the cuve $y = f''(x)$ or as the rate of change of $f''(x)$. If $y = f(x)$, then alternative notations for the third derivative are
	$$y''' = f'''(x) = \dfrac{d}{dx} \Big( \dfrac{d^2 y}{dx^2} \Big) = \dfrac{d^3 y}{dx^3}$$
	The process can be continued. The fourth derivative $f''''$ is usually denoted by $f^{(4)}$. In general, the $n$th derivative of $f$ is denoted by $f^{(n)}$ and is obtained from $f$ by differentiating $n$ times. If $y = f(x)$, we write
	$$y^{(n)} = f^{(n)}(x) = \dfrac{d^n y}{dx^n}$$
	The third derivative of the position function (the derivative of the acceleration function) is called the \textbf{jerk} and is the rate of change of acceleration.
	$$j = \dfrac{da}{dt} = \dfrac{d^3 s}{dt^3}$$
	\\
	\hline
\end{tabularx}}
\vspace{12pt}

\end{center}
\end{document}
