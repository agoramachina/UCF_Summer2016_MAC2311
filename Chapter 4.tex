\documentclass{article}
\usepackage{mathtools, geometry, amssymb, booktabs, array, tabularx, enumitem, pgfplots, tikz}
\usetikzlibrary{decorations.markings, calc}
\pgfplotsset{compat=1.9}
\newcommand{\tabitem}{~~\llap{\textbullet}~~}
\geometry{portrait, margin=1in}
\setlist{nosep}
\pagestyle{empty}
\begin{document}

\Huge\textbf{Chapter 4}
\vspace{16pt}

\begin{center}
\Large\textbf{4.1 Maximum and Minimum Values}

\noindent\hfill\rule{0.3\textwidth}{.4pt}\hfill
\vspace{24pt}

\large
\def\arraystretch{1.3}
{\setlength{\tabcolsep}{16pt}
\begin{tabularx}{.9\textwidth}{|X|}
\hline
	Let $c$ be a number in the domain $D$ of a function $f$. Then $f(c)$ is the \\
	\hspace{12pt} \tabitem \textbf{absolute maximum} value of $f$ on $D$ if $f(c) \geq f(x)$ for all $x$ in $D$. \\
	\hspace{12pt} \tabitem \textbf{absolute minimum} value of $f$ on $D$ if $f(c) \leq f(x)$ for all $x$ in $D$. \\

\hline
\end{tabularx}}
\end{center}

\large
\vspace{5pt}
\begin{itemize}[leftmargin=5pt]
	\item An absolute maximum or minimum is also called a \textbf{global} maximum or minimum. 
	\item The maximum and minimum values of $f$ are called \textbf{extreme values} of $f$. 
\end{itemize}
\vspace{12pt}

\begin{center}
\large
\def\arraystretch{1.3}
{\setlength{\tabcolsep}{16pt}
\begin{tabularx}{.9\textwidth}{|X|}
\hline
	The number $f(c)$ is a \\
	\hspace{12pt} \tabitem \textbf{local maximum} value of $f$ if $f(c) \geq f(x)$ when $x$ is near $c$. \\
	\hspace{12pt} \tabitem \textbf{local minimum} value of $f$ if $f(c) \leq f(x)$ when $x$ is near $c$. \\

\hline
\end{tabularx}}
\end{center}

\large
\vspace{5pt}
\begin{itemize}[leftmargin=5pt]
	\item If we say that something is true \textbf{near} $c$, we mean that it is true on some \textbf{open interval} containing $c$.
\end{itemize}
\vspace{12pt}

\begin{center}
\large
\def\arraystretch{1.3}
{\setlength{\tabcolsep}{16pt}
\begin{tabularx}{.9\textwidth}{|X|}
\hline
	\textbf{The Extreme Value Theorem} \: If $f$ is continuous on a closed interval $[a,b]$, then $f$ attains an absolute maximum value $f(c)$ and an absolute minimum value $f(d)$ at some numbers $c$ and $d$ in $[a,b]$.\\
\hline
\end{tabularx}}
\end{center}

\large
\vspace{5pt}
\begin{itemize}[leftmargin=5pt]
	\item An extreme value can be taken on more than once.
	\item A function need not posess extreme values if either hypothesis (continuity or closed interval) is omitted from the Extreme Value Theorem.
\end{itemize}
\vspace{12pt}

\begin{center}
\large
\def\arraystretch{1.3}
{\setlength{\tabcolsep}{16pt}
\begin{tabularx}{.9\textwidth}{|X|}
\hline
	\textbf{Fermat's Theorem} \: If $f$ has a local maximum or minimum at $c$, and if $f'(c)$ exists, then $f'(c) = 0$.\\
\hline
\end{tabularx}}
\end{center}

\large
\vspace{5pt}
\begin{itemize}[leftmargin=5pt]
	\item Even when $f'(c) = 0$ there need not be a maximum or minimum at $c$. Furthermore, there may be an extreme value even when $f'(c)$ does not exist.  Such numbers are called \textbf{critical numbers}.
\end{itemize}
\vspace{12pt}

\begin{center}
\large
\def\arraystretch{1.3}
{\setlength{\tabcolsep}{16pt}
\begin{tabularx}{.9\textwidth}{|X|}
\hline
	A \textbf{critical number} of a function $f$ is a number $c$ in the domain of $f$ such that either $f'(c) = 0$ or $f'(c)$ does not exist. \\
\hline
\end{tabularx}}
\end{center}

\large
\vspace{5pt}
\begin{itemize}[leftmargin=5pt]
	\item If $f$ has a local maximum or minimum at $c$, then $c$ is a critical number of $f$.
\end{itemize}
\vspace{12pt}
\pagebreak

\begin{center}
\large
\def\arraystretch{1.3}
{\setlength{\tabcolsep}{16pt}
\begin{tabularx}{.9\textwidth}{|X|}
\hline
	\vspace{5pt}
	\textbf{The Closed Interval Method} \: To find the \textit{absolute} maximum and minimum values of a continuous function $f$ on a closed interval $[a,b]$: \\
	\begin{enumerate}[itemsep=5pt]
	\item Find the values of $f$ at the \textit{critical numbers} of $f$ in $(a,b)$.
	\item Find the values of $f$ at the \textit{endpoints} of the interval.
	\item The largest of the values from Steps 1 and 2 is the absolute maximum value; the smallest of these values is the absolute minimum value.
	\end{enumerate} \\
	\hline
\end{tabularx}}
\end{center}
\vspace{24pt}

\begin{center}
\Large\textbf{4.2 The Mean Value Theorem}

\noindent\hfill\rule{0.3\textwidth}{.4pt}\hfill
\vspace{20pt}

\begin{center}
\large
\def\arraystretch{1.3}
{\setlength{\tabcolsep}{16pt}
\begin{tabularx}{.9\textwidth}{|X|}
\hline
	\vspace{5pt}
	\textbf{The Mean Value Theorem} \: Let $f$ be a function that satisfies the following hypotheses: 
	\vspace{10pt}
	\begin{enumerate}[itemsep=5pt]
	\item $f$ is continuous on the closed interval $[a,b]$.
	\item $f$ is differentiable on the open interval $(a,b)$.
	\end{enumerate} 
	\vspace{10pt}
	Then there is a number $c$ in $(a,b)$ such that
	\vspace{5pt}
	\begin{center}
	$f'(c) = \dfrac{f(b)-f(a)}{b-a}$
	\end{center}
	or, equivalently,
	\begin{center}
	$f(b)-f(a) = f'(c)(b-a)$
	\end{center}
	\\[5pt]
	\hline
\end{tabularx}}
\end{center}

\large
\vspace{16pt}
\begin{itemize}[leftmargin=5pt]
	\item Recall that the slope of the secant line $AB$ of points $A(a,f(a))$ and $B(b,f(b))$ is \\[3pt]
	$m_{AB} = \dfrac{f(b)-f(a)}{b-a}$ and $f'(c)$ is the slope of the tangent line at the point $(c, f(c))$.
\end{itemize}
\vspace{16pt}

\begin{center}
\large
\def\arraystretch{1.3}
{\setlength{\tabcolsep}{16pt}
\begin{tabularx}{.9\textwidth}{|X|}
\hline
	\vspace{.5pt}
	\textbf{Theorem} \: If $f'(x) = 0$ for all $x$ in an interval $(a,b)$, then $f$ is constant on $(a,b)$. \\
	\vspace{.5pt}
	\textbf{Corollary} \: If $f'(x) = g'(x)$ for all $x$ in an interval $(a,b)$, then $f-g$ is constant on $(a,b)$; that is, $f(x) = g(x) + c$ where $c$ is a constant. \\
\vspace{.5pt}\\
\hline
\end{tabularx}}
\end{center}
\end{center}
\pagebreak

\begin{center}
\Large\textbf{4.3 How Derivatives Affect the Shape of a Graph}

\noindent\hfill\rule{0.3\textwidth}{.4pt}\hfill
\vspace{24pt}

\large
\def\arraystretch{1.3}
{\setlength{\tabcolsep}{16pt}
\begin{tabularx}{.9\textwidth}{|X|}
\hline
	\textbf{Increasing/Decreasing Test} \\
	(a) If $f'(x) > 0$ on an interval, then $f$ is \textit{increasing} on that interval. \\
	(b) If $f'(x) < 0$ on an interval, then $f$ is \textit{decreasing} on that interval. \\

\hline
\end{tabularx}}
\vspace{16pt}

\large
\def\arraystretch{1.3}
{\setlength{\tabcolsep}{16pt}
\begin{tabularx}{.9\textwidth}{|X|}
\hline
	\textbf{The First Derivative Test} \: Suppose that $c$ is a critical number of a continuous function $f$. \\
	
	(a) If $f'$ changes from positive to negative at $c$, then $f$ has a local maximum \\ \hspace{16pt} at $c$. \\
	(b) If $f'$ changes from negative to positive at $c$, then $f$ has a local minimum \\ \hspace{16pt} at $c$. \\
	(c) If $f'$ does not change sign at $c$ (for example, if $f'$ is positive or negative \\ \hspace{16pt} on both sides of $c$), then $f$ has no local maximum or minimum at $c$. \\  
\hline
\end{tabularx}}
\vspace{16pt}
\end{center}
%TODO: add example graphs from p 291, 293

\begin{center}
\large
\def\arraystretch{1.3}
{\setlength{\tabcolsep}{16pt}
\begin{tabularx}{.9\textwidth}{|X|}
\hline
	\textbf{Definition of Concavity} \: If the graph of $f$ lies above all of its tangents on an interval $I$, then it is called \textbf{concave upward} on $I$. If the graph of $f$ lies below all of its tangents on $I$, it is called \textbf{concave downward} on $I$. \\
\hline
\end{tabularx}}
\vspace{16pt}

\large
\def\arraystretch{1.3}
{\setlength{\tabcolsep}{16pt}
\begin{tabularx}{.9\textwidth}{|X|}
\hline
	\textbf{Concavity Test} \\
	(a) If $f"(x) > 0$ for all $x$ in $I$, then the graph of $f$ is concave upward on $I$. \\
	(b) If $f"(x) < 0$ for all $x$ in $I$, then the graph of $f$ is concave downward \\ \hspace{16pt} on $I$. \\
\hline
\end{tabularx}}
\vspace{16pt}

\large
\def\arraystretch{1.3}
{\setlength{\tabcolsep}{16pt}
\begin{tabularx}{.9\textwidth}{|X|}
\hline
	\textbf{Definition of the Inflection Point} \: A point $P$ on a curve $y=f(x)$ is called an \textbf{inflection point} if $f$ is continuous there and the curve changes from concave upward to concave downward or from concave downward to concave upward at $P$. \\
\hline
\end{tabularx}}
\vspace{16pt}

\large
\def\arraystretch{1.3}
{\setlength{\tabcolsep}{16pt}
\begin{tabularx}{.9\textwidth}{|X|}
\hline
	\textbf{The Second Derivative Test} \: Suppose $f"$ is continuous near $c$. \\
	(a) If $f'(c) = 0$ and $f"(c) > 0$, then $f$ has a local minimum at $c$. \\
	(b) If $f'(c) = 0$ and $f"(c) < 0$, then $f$ has a local maximum at $c$. \\
\hline
\end{tabularx}}
\vspace{16pt}
\end{center}
\pagebreak


\end{document}

